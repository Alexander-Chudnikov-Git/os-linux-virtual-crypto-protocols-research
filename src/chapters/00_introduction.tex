\unnumsection{Введение}
Выпускная квалификационная работа посвящена исследованию и практической реализации виртуального криптографического сетевого интерфейса в пространстве ядра операционной системы Linux с использованием отечественных криптографических стандартов.

\unnumsubsection{Актуальность темы исследования}
В связи с геополитическими изменениями происходящими в мире, кибератаки стали неотъемлемым инструментом в руках недружественных государств.
Обеспечение безопасности в информационном пространстве становится всё более и более сложным.
В частности это происходит из-за повсеместного использования криптографических стандартов и программного обеспечения, разработанного в недружественных странах.
Такие стандарты и ПО могут спокойно содержать в себе как программные закладки, так и незадокументированные возможности, как например в генераторе псевдослучайных чисел Dual EC.~\cite{dual_ec_backdoor}
Единственным адекватным решением от такого роду угроз, становится использование отечественных криптографических стандартов.\\

Большой проблемой остается отсутствие каких-либо отечественных аналогов зарубежным протоколам позволяющим построить виртуальные частные сети, таких как OpenVPN или WireGuard.
В данной работе будет рассматриваться именно WireGuard, так как его реализация значительно проще аналогов.

\unnumsubsection{Объект исследования}
Объектом исследования в данной работе выступает ядро операционной системы Linux, сетевой стек ядра, криптографический API (Crypto API), а также механизмы передачи информации из пространства пользователя в ядро, такие как протокол Netlink.

\unnumsubsection{Предмет исследования}
Предметом исследования выступают конкретные механизмы, архитектурные решения и программные интерфейсы, используемые в WireGuard, а также методы интеграции в нее отечественных криптоалгоритмов и протоколов.

\unnumsubsection{Степень научной разработанности темы}
Вопросы реализации сетевых подсистем и криптографических API в ядре Linux широко освещены в трудах разработчиков ядра и в технической документации.
Архитектура WireGuard подробно описана его создателем Джейсоном Доненфилдом, а так же в работе <<Analysis of the WireGuard
protocol>> Питера Ву.~\cite{wireguard_analysis}
Вместе с тем уже предпринимались попытки адаптации протокола WireGuard для использования отечественных криптоалгоритмов, в частности, в рамках проекта <<RuWireGuard>>.
Однако существующие реализации данного протокола, как правило, выполнены на языке Go и функционируют в пространстве пользователя (userspace).
Такой подход неизбежно приводит к значительным потерям производительности из-за необходимости постоянного копирования данных между ядром и пространством пользователя, а также из-за накладных расходов на переключение контекста.
Это делает подобные решения малоэффективными для высоконагруженных систем и подтверждает наличие пробела в исследованиях, а именно — отсутствие полноценной реализации в пространстве ядра Linux.

\unnumsubsection{Цель дипломной работы}
Целью дипломной работы является исследование возможности встраивания в ядро Linux виртульного криптоинтерфейса поддерживающего отечественные криптографические стандарты.

С целью упрощения работы над данной работой, было принято решение разделить её на следующие этапы:
\begin{enumerate}
    \item Анализ архитектуры и принципов работы виртуальных сетевых интерфейсов в ядре операционной системы Linux.
    \item Исследование криптографических подсистем ядра Linux и их API на предмет возможности интеграции отечественных алгоритмов.
    \item Изучение механизмов взаимодействия и передачи управляющей информации из пространства пользователя в ядро, в частности, протокол Netlink.
    \item Проведение анализа и обоснование выбора транспортного криптографического протокола для создаваемого решения.
    \item Разработка прототипа виртуального криптоинтерфейса, основываясь на архитектуре и принципах работы WireGuard.
\end{enumerate}

\unnumsubsection{Методологическая основа исследования}
Теоретическую основу исследования составили системный анализ, изучение и обобщение технической документации ядра Linux, спецификаций протоколов, а также анализ исходного кода существующих решений.
Практическая часть основывается на программирован для ядра Linux и объектно-ориентированного проектирования.

Для оценки производительности разработанного прототипа будет использована следующая методика проведения экспериментов:
\begin{enumerate}
    \item \textbf{Создание тестового стенда}, состоящего из двух физических или виртуальных машин (клиент и сервер), соединенных сетевым каналом.
    \item \textbf{Измерение базовых показателей канала} без шифрования с помощью утилит `iperf3` (для определения максимальной пропускной способности) и `ping` (для определения времени задержки).
    \item \textbf{Развертывание и настройка разработанного прототипа} на тестовом стенде для создания защищенного туннеля.
    \item \textbf{Проведение аналогичных измерений} пропускной способности и времени задержки через созданный криптографический туннель.
    \item \textbf{Сравнение результатов} с показателями эталонного userspace-решения (например, реализации RuWireGuard на Go), с базовыми показателями оригинального WireGuard и с базовыми показателями канала для оценки накладных расходов и подтверждения эффективности реализации на уровне ядра.
\end{enumerate}

\unnumsubsection{Научная новизна}
Научная новизна работы заключается в исследовании и предложении подхода к интеграции отечественных криптографических алгоритмов в архитектуру WireGuard на уровне ядра ОС.
Все предидущие подходы строились на основе использования пользовательского пространства, в данной же работе предложен другой подход к решению данной задачи, который в теории должен обеспечить более высокую производительность и меньшие ресурсоемкость, чем существующие решения.

\unnumsubsection{Структура работы}
Дипломная работа состоит из введения, трех глав, заключения и списка использованных источников.
Во введении обосновывается актуальность темы, определяются объект, предмет, цель и задачи исследования.
В первой главе рассматриваются теоретические основы сетевых и криптографических подсистем ядра Linux, а также технологий и протоколов которые будут затронуты в данной работе.
Вторая глава посвящена анализу архитектуры ядра OS Linux, архитектуры WireGuard, а также анализу существующих решений на базе отечественных стандартов.
В третьей главе описывается процесс разработки прототипа, а также представлены результаты экспериментальной оценки его производительности.
В заключении подводятся итоги проделанной работы и формулируются основные выводы.
