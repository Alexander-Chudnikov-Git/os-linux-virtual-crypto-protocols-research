\begin{figure}[h!]
\centering
\begin{tikzpicture}[
    auto,
    node distance = 2cm and 2.5cm,
    block/.style = {
        rectangle,
        draw,
        minimum height=2.5em,
        text width=6.5em,
        text centered,
        font=\sffamily
    },
    container/.style = {
        rectangle,
        draw,
        inner sep=10pt
    }
]
    \node[block, fill=teal!20] (interface) {\textbf{Сетевой интерфейс}};
    \node[block, fill=teal!20, right=of interface] (device) {\textbf{Устройство}};
    \node[block, fill=teal!20, text width=7em, right=of device] (adapter) {\textbf{Сетевой адаптер}};

    \node[container, fill=teal!15, fit=(device), inner ysep=15pt, inner xsep=50pt, yshift=10pt] (driver_box) {};
    \node[container, fill=teal!10, fit=(interface) (driver_box)] (linux_box) {};
    \node[container, fill=teal!15, fit=(device), inner ysep=15pt, inner xsep=50pt, yshift=10pt] (driver_box) {};

    \node[block, fill=teal!20] (interface) {\textbf{Сетевой интерфейс}};
    \node[block, fill=teal!20, right=of interface] (device) {\textbf{Устройство}};
    \node[block, fill=teal!20, text width=7em, right=of device] (adapter) {\textbf{Сетевой адаптер}};

    \node[anchor=north, yshift=-2pt] at (driver_box.north) {\textbf{Драйвер сетевого устройства}};
    \node[anchor=north west, inner sep=5pt] at (linux_box.north west) {\textbf{Ядро Linux}};

    \draw[-Latex] (interface.east) -- (device.west);
    \draw[-Latex] (device.east) -- (adapter.west);
\end{tikzpicture}
\caption{Сетевой интерфейс в Linux.}
\label{fig:linux-net-stack}
\end{figure}
