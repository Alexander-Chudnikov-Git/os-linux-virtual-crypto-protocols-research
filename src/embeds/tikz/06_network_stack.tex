\begin{figure}[h!]
\centering
\begin{tikzpicture}[
    node distance=1cm and 0.1cm,
    layer_box/.style={rectangle, draw, fill=teal!20,
                      minimum width=4.1cm, minimum height=1cm,
                      font=\footnotesize\bfseries},
    data_block/.style={rectangle, draw, fill=blue!20,
                       minimum width=1.8cm, text height=1.5ex, text depth=0.25ex, font=\footnotesize\bfseries},
    header_block/.style={rectangle, draw, fill=orange!30,
                         minimum width=1.0cm, text height=1.5ex, text depth=0.25ex,
                         font=\footnotesize},
    arrow_style/.style={-Stealth[], thick, shorten >=2pt, shorten <=2pt}
]

\draw[black, dot diameter=2pt, dot spacing=5pt, dots] (-8,0)  -- (8,0);
\draw[black, dot diameter=2pt, dot spacing=5pt, dots] (-8,-2) -- (8,-2);
\draw[black, dot diameter=2pt, dot spacing=5pt, dots] (-8,-4) -- (8,-4);
\draw[black, dot diameter=2pt, dot spacing=5pt, dots] (-8,-6) -- (8,-6);
\draw[black, dot diameter=2pt, dot spacing=5pt, dots] (-8,-8) -- (8,-8);

% ===================================================================
% Шаг 1: Рисуем центральную колонку с названиями уровней
% ===================================================================
\node (app) at (0,-1) [layer_box] {Прикладной уровень};
\node (trn)  [layer_box, below=of app] {Транспортный уровень};
\node (net)  [layer_box, below=of trn] {Сетевой уровень};
\node (link) [layer_box, below=of net] {Канальный уровень};

% ===================================================================
% Шаг 2: Рисуем левую колонку - ИНКАПСУЛЯЦИЯ (отправка)
% ===================================================================
\node [above=0.5cm of app, xshift=-3.5cm] {\textbf{Инкапсуляция (отправка)}};

% Уровень приложения
\node (encap_app_data) [data_block, left=of app] {Данные};

% Транспортный уровень
\node (encap_trn_data) [data_block, left=of trn] {Данные};
\node (encap_trn_hdr)  [header_block, left=of encap_trn_data, node distance=0] {TCP/UDP};

% Сетевой уровень
\node (encap_net_data) [data_block, left=of net] {Данные};
\node (encap_net_hdr1) [header_block, left=of encap_net_data, node distance=0] {TCP/UDP};
\node (encap_net_hdr2) [header_block, left=of encap_net_hdr1, node distance=0] {IP};

% Канальный уровень
\node (encap_link_data) [data_block, left=of link] {Данные};
\node (encap_link_hdr1) [header_block, left=of encap_link_data, node distance=0] {TCP/UDP};
\node (encap_link_hdr2) [header_block, left=of encap_link_hdr1, node distance=0] {IP};
\node (encap_link_hdr3) [header_block, left=of encap_link_hdr2, node distance=0] {Кадр};

% Стрелки вниз
\draw [arrow_style] (encap_app_data.south) -- (encap_trn_data.north);
\draw [arrow_style] (encap_trn_data.south) -- (encap_net_data.north);
\draw [arrow_style] (encap_net_data.south) -- (encap_link_data.north);

% ===================================================================
% Шаг 3: Рисуем правую колонку - ДЕКАПСУЛЯЦИЯ (получение)
% ===================================================================
\node [above=0.5cm of app, xshift=3.5cm] {\textbf{Декапсуляция (получение)}};

% Канальный уровень
\node (decap_link_data) [data_block, right=of link] {Данные};
\node (decap_link_hdr1) [header_block, right=of decap_link_data, node distance=0] {TCP/UDP};
\node (decap_link_hdr2) [header_block, right=of decap_link_hdr1, node distance=0] {IP};
\node (decap_link_hdr3) [header_block, right=of decap_link_hdr2, node distance=0] {Кадр};

% Сетевой уровень
\node (decap_net_data) [data_block, right=of net] {Данные};
\node (decap_net_hdr1) [header_block, right=of decap_net_data, node distance=0] {TCP/UDP};
\node (decap_net_hdr2) [header_block, right=of decap_net_hdr1, node distance=0] {IP};

% Транспортный уровень
\node (decap_trn_data) [data_block, right=of trn] {Данные};
\node (decap_trn_hdr)  [header_block, right=of decap_trn_data, node distance=0] {TCP/UDP};

% Уровень приложения
\node (decap_app_data) [data_block, right=of app] {Данные};

% Стрелки вверх
\draw [arrow_style] (decap_link_data.north) -- (decap_net_data.south);
\draw [arrow_style] (decap_net_data.north) -- (decap_trn_data.south);
\draw [arrow_style] (decap_trn_data.north) -- (decap_app_data.south);

\end{tikzpicture}
\caption{Процессы инкапсуляции и декапсуляции данных в модели TCP/IP}
\label{fig:encap_decap}
\end{figure}
