\begin{abstract}
\noindent\textbf{Тема:}
Исследование реализации виртуальных криптографических сетевых интерфейсов в ядре ОС Linux.

\vspace{\baselineskip}

\noindent\textbf{Цель работы:}
Целью данной работы является исследование механизмов ядра операционной системы Linux с целью последующей реализации виртуальных сетевых интерфейсов на базе криптографических протоколов входящих в ГОСТ.

\vspace{\baselineskip}

\noindent\textbf{Поставленные задачи:}
\begin{itemize}
    \item Исследование виртуальных сетевых интерфейсов в ядре Linux.
    \item Анализ криптографической подсистемы ядра (\texttt{Crypto API}).
    \item Изучение механизма передачи ключевой информации из пространства пользователя в ядро (\texttt{netlink}).
    \item Выбор транспортного криптографического протокола для защищенной передачи данных.
    \item Разработка прототипа виртуального криптоинтерфейса на примере интерфейса \texttt{Wireguard}.
\end{itemize}

\vspace{\baselineskip}

\noindent\textbf{Полученные результаты:}
%В ходе работы был проведен комплексный анализ подсистем ядра Linux, отвечающих за сетевое взаимодействие и шифрование.
%Был успешно разработан прототип виртуального криптоинтерфейса, который демонстрирует возможность интеграции криптографических операций непосредственно на уровне ядра для повышения производительности и безопасности.
%Прототип использует \texttt{netlink} для конфигурации и управления, а его архитектура основана на принципах, заложенных в \texttt{Wireguard}.

\vspace{\baselineskip}

\noindent\textbf{Предложенные рекомендации:}
%Полученные результаты и разработанный прототип могут служить основой для создания кастомных высокопроизводительных VPN-решений и других средств защиты каналов связи.
%Рекомендуется дальнейшее развитие проекта с целью расширения поддержки различных криптографических алгоритмов и улучшения механизмов управления для упрощения интеграции в сложные сетевые инфраструктуры.
\end{abstract}

\newpage

\begin{otherlanguage}{english}
\begin{abstract}
\noindent\textbf{Topic:}
Research on the implementation of virtual cryptographic network interfaces in the Linux OS kernel.

\vspace{\baselineskip}

\noindent\textbf{Purpose:}
The purpose of this work is to study the mechanisms of the Linux operating system kernel for the subsequent implementation of virtual network interfaces based on cryptographic protocols included in the GOST standard.

\vspace{\baselineskip}

\noindent\textbf{Tasks:}
\begin{itemize}
\item Research of virtual network interfaces in the Linux kernel.
\item Analysis of the kernel's cryptographic subsystem (\texttt{Crypto API}).
\item Study of the mechanism for transferring key information from userspace to the kernel (\texttt{netlink}).
\item Selection of a transport cryptographic protocol for secure data transmission.
\item Development of a prototype of a virtual crypto-interface based on the example of the \texttt{Wireguard} interface.
\end{itemize}

\vspace{\baselineskip}

\noindent\textbf{Results:}
%In the course of this work, a comprehensive analysis of the Linux kernel subsystems responsible for networking and encryption was performed.
%A prototype of a virtual crypto-interface was successfully developed, which demonstrates the ability to integrate cryptographic operations directly at the kernel level to improve performance and security.
%The prototype uses \texttt{netlink} for configuration and management, and its architecture is based on the principles embedded in \texttt{Wireguard}.

\vspace{\baselineskip}

\noindent\textbf{Recommendations:}
%The obtained results and the developed prototype can serve as a basis for creating custom high-performance VPN solutions and other means of protecting communication channels.
%Further development of the project is recommended to expand support for various cryptographic algorithms and improve management mechanisms to simplify integration into complex network infrastructures.
\end{abstract}
\end{otherlanguage}

\newpage
